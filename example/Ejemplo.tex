\documentclass[aspectratio=169,usenames,dvipsnames,svgnames,table]{beamer}
\usepackage{presentationUNAM}
\ThemeDarkBlue
\lstset{language=Java}

\addbibresource{icc_pres.bib}

% Información estándar
%\title{Ejemplo}
\title{Ejemplo con un título superlargo para demostrar cómo funciona incluso con cambio de línea.}
\subtitle{Presentación con subítitulos}
\author[Ver\'onica E. Arriola-Rios]{Verónica E. Arriola-Rios}
\institute{Facultad de Ciencias, UNAM}
\date{\today}
%\titlegraphic{\includegraphics{<logo>}}
\titlegraphic{FC}



\begin{document}

%% Título
\frame{\titlepage}

\section[SLarga]{Sección también con un título superlargo para probar que también funcione correctamente.}

\section{Lo básico}

\subsection{Ambientes}

\subsubsection{Definiciones}

\begin{frame}
\begin{definition}[Definición]
 Una \emph{definición} establece los elementos que conforman...
\end{definition}

También tenemos ecuaciones con una fuente diferente:
\begin{align}
 eq &= \sum_{i} x_i \\
 \oint f(x) dx &= 0
\end{align}

Cita: \cite{Viso2012}

\end{frame}


\subsubsection{Listados}

\begin{frame}{Un cuadro con...}
\framesubtitle{¡Subtítulo!}
Probemos ahora las viñetas:
\begin{itemize}
 \item Primer elemento.
 \begin{itemize}
  \item Elemento anidado.
  \begin{itemize}
   \item Otro más.
   \item Uno de adorno.
  \end{itemize}
 \end{itemize}
\end{itemize}

Y las enumeraciones:
\begin{enumerate}
 \item Primer elemento con un texto un poco largo para probar.
 \begin{enumerate}
  \item Elemento anidado, con más royo para ocupar toda la línea.
  \begin{enumerate}
   \item Otro más.
   \item Uno de adorno.
  \end{enumerate}
 \end{enumerate}
\end{enumerate}

\end{frame}



\section{Para programadores}
\subsection{Algoritmo}

\begin{frame}{Algoritmo}
\begin{algorithm}[H]
\caption{Búsqueda en un árbol.}\label{alg:busqueda_general}
\begin{algorithmic}[1]
  \Function{BúsquedaEnÁrbol}{$problema$, $estrategia$}
   \State $margen \leftarrow \{$ new \Call{NodoBúsqueda}{$problema.s_i$}$\}$
   \Loop
    \If{\Call{vacía}{$margen$}} \Return Fallo \EndIf
    \State $nodo \leftarrow$ seleccionaDe($margen$, $estrategia$)
    \If{$problema$.PruebaEsMeta($nodo.estado$)}
     \State \Return caminoA($nodo$)
    \EndIf
    \State $margen \leftarrow margen$ + expande($problema$, $nodo$) \label{alg:estrategia}
   \EndLoop
  \EndFunction
\end{algorithmic}
\end{algorithm}
\end{frame}



\subsection{Código}

\begin{frame}[fragile]{Código}
\begin{lstlisting}
public static <C extends Comparable<? super C>>
       void bubleSort(C[] a) {
  boolean swapped = false;
  for(int i = 0; i < a.length - 1; i++) {
    swapped = false;
    for(int j = 1; j < a.length - i; j++) {
      if(a[j-1].compareTo(a[j]) > 0) {
        swap(a, j-1,j);
        swapped = true;
      }
    }
    if(!swapped) return;
  }
}
\end{lstlisting}
\end{frame}




\section{Diagramas con tikz}

\begin{frame}{Computadora}
\centering
 \begin{tikzpicture}[>=stealth, very thick,
   component/.style={
     rectangle,                     % forma del nodo
     very thick,                    % estilo del borde
     draw=pink,                     % color del borde
     top color=white,               % relleno sombreado blanco en la parte de arriba...
     bottom color=pink!50!black!20, % y otro color en la parte de abajo
     align = center
   }]
  \node [component] (mem) {Memoria};
  \node [component, text width = 6em, below = of mem] (prc) {Procesador central};
  \node [component, text width = 6em, left = of prc] (i) {Elementos de entrada};
  \node [component, text width = 6em, right = of prc] (o) {Elementos de salida};
  \draw [->] (i) -- (prc);
  \draw [->] (prc) -- (o);
  \draw [<->] (prc) -- (mem);
 \end{tikzpicture}
 \begin{center}
  Diagrama de bloques de una computadora.
 \end{center}

\end{frame}



\begin{frame}{Columnas}
\begin{columns}
\column{0.5\textwidth}

\begin{itemize}
 \item Matemático húngaro-estadounidense
 \item Con contribuciones en:
 \begin{itemize}
  \item Uno,
  \item dos,
  \item tres.
 \end{itemize}
\end{itemize}

\column{0.5\textwidth}
 
 Y otro texto en la columna dos.
\end{columns}
\end{frame}




\section{Bibliografía}
\stepcounter{subsection}
\begin{frame}[allowframebreaks]{Bibliografía}
Manual:
\begin{thebibliography}{9}
\setbeamertemplate{bibliography item}[online]
\bibitem{A} ItemA
\setbeamertemplate{bibliography item}[book]
\bibitem{B} ItemB
\setbeamertemplate{bibliography item}[article]
\bibitem{C} ItemC
\setbeamertemplate{bibliography item}[triangle]
\bibitem{D} ItemD
\setbeamertemplate{bibliography item}[text]
\bibitem{E} ItemE
\end{thebibliography}

\vspace*{2ex}
Usando biblatex:
\printbibliography
\end{frame}

\end{document}
